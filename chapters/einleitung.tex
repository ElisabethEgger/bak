\chapter{Einleitung}
\label{cha:Einleitung}

\section{Ausgangssituation}
TODO ÜBERARBEITUNG
\vspace{\baselineskip}
\vspace{\baselineskip}

Von der Firma Siemens wurde neben Elektrifizierung und Automatisierung die Digitalisierung als dritter Punkt in der "`Vision 2020 -- Strategie im Überblick"' festgelegt \parencite{SiemensVision2020}. Mit 7 - 9\% prognostiziertem Wachstum hat die Digitalisierung dabei den größten Anteil an den definierten Zielen. Siemens baut seit 2016 eine Cloud-Infrastruktur auf, die es den Gerätekomponenten ermöglicht die Daten zu speichern. 



\section{Motivation}
TODO

Connect - gelöst - wird verkauft - Daten werden gesammelt
Cloud - gelöst - wird verkauft - Daten werden übertragen und gespeichert
Apps - bis auf FleetManager und ManageMyMachine noch nicht vorhanden - Daten werden noch nicht wirklich ausgelesen bzw. analysiert und visualisiert.


\section{Zielsetzung}
TODO
Evaluierung, wie sich die Plattform MindSphere für die Umsetzung von Kundenprojekten eignet. Konkret soll eine Applikation umgesetzt werden, welche ein Alarming-System bereitstellt. 


\section{Aufbau der Arbeit}
TODO ÜBERARBEITUNG
\vspace{\baselineskip}
\vspace{\baselineskip}

Die Arbeit gliedert sich in fünf Kapitel. 

Das erste Kapitel beschreibt die Ausgangssituation, Motivation und Zielsetzung sowie den Aufbau der Arbeit.

Im zweiten Kapitel werden die Grundlagen für diese Arbeit erläutert. Beginnend mit einem geschichtlichen Rückblick in der Entwicklung der Industrialisierung und einer Erklärung des Begriffs Industrie 4.0 samt Erläuterung von den Zielen und Standards bei Industrie 4.0 werden auch die Themen Cloud Computing und Internet of Things behandelt. Zusätzlich wird das Projekt Siemens MindSphere beschrieben. Die Motivation für Siemens MindSphere, deren Aufbau, die Stärken aber auch die technischen Einschränkungen und ein Überblick über das MindSphere-Preismodell schließen dieses Kapitel ab.

Kapitel 3 beschreibt die Realisierung der Aufgabenstellung dieser Arbeit. Die Realisierung gliedert sich in System-Architektur, Architektur und Design, ausgewählte Implementierungsaspekte und nicht funktionale Merkmale.

Im 4. Kapitel werden die Ergebnisse der Arbeit dokumentiert und diskutiert. 

Kapitel 5 enthält eine Zusammenfassung dieser Arbeit und gibt einen Ausblick in Richtung möglicher Weiterentwicklung der Thematik.


