\chapter{Kurzfassung}


TODO ÜBERARBEITUNG
\vspace{\baselineskip}
\vspace{\baselineskip}


Die Fa. Siemens AG bietet mit MindSphere eine neue Plattform für den Einsatz von Industrial Internet of Things und Cloud-Computing an. Industrielle Anlagen können so auf einfache Weise Daten in eine Cloud-Umgebung liefern. 

Im Detail geschieht dies durch Sensoren, welche Daten über speicherprogrammierbare Steuerungen sammeln und diese mit Hilfe von MindConnect Modulen an die Cloud MindSphere transferieren, wo diese gespeichert werden. 

Von diesem Zeitpunkt an stehen die Daten für diverse Auswertungen über die MindSphere API jederzeit zur Verfügung. 

Am Standort Siemens Graz soll nun evaluiert werden, wie sich die Plattform MindSphere für die Umsetzung von Kundenprojekten eignet. 

Ziel dieser Bachelorarbeit ist es, zu zeigen, wie eine Umsetzung einer MindSphere Applikation funktionieren kann und zu prüfen, welche Probleme dabei derzeit noch auftreten. Konkret soll ein einfaches Beispiel implementiert werden. 

Eine Testumgebung bestehend aus diversen (simulierten) Sensoren, einem MindConnect Modul und einem Entwicklerkonto für die MindSphere Cloud Plattform werden von der Fa. Siemens AG zur Verfügung gestellt. Konkretes Ziel ist es, ein einfaches Alarming-System zu implementieren, welches einen Kunden informiert, sobald die Werte eines Sensors einen fix vorgegebenen Wert über- oder unterschreiten. Für die Benachrichtigung wird eine Webapplikation in HTML5 / eine Java Desktop Applikation gewählt, welche die derzeit aktiven Warnungen anzeigt.
