\chapter{Kurzfassung}

Die Fa. Siemens AG bietet mit MindSphere eine neue Plattform für den Einsatz von Industrial Internet of Things und Cloud-Computing. Industrielle Anlagen können so auf einfache Weise Daten in eine Cloud-Umgebung liefern. Sensoren sammeln Daten und transferieren diese mit Hilfe von MindConnect Modulen an die Cloud MindSphere, wo diese gespeichert werden. Von diesem Zeitpunkt an stehen diese Daten den Kunden für diverse Auswertungen über die MindSphere API zur Verfügung. Am Standort Graz soll nun evaluiert werden, wie sich die Plattform für die Umsetzung von Kundenprojekten eignet. 

Ziel dieser Bachelorarbeit ist es zu prüfen und zu zeigen, welche Probleme bei der Umsetzung von Projekten mit MindSphere derzeit noch vorhanden sind. Weiters soll ein einfaches Beispiel implementiert werden. Eine Testumgebung bestehend aus diversen (simulierten) Sensoren, MindSphere Connect und einem Account für die MindSphere Cloud Plattform werden von der Fa. Siemens AG zur Verfügung gestellt. Konkret soll ein einfaches Alarming-System implementiert werden, welches einen Kunden informiert, sobald die Werte eines Sensors einen fix vorgegebenen Wert über- oder unterschreiten. Für die Benachrichtigung wird eine Webapplikation in HTML5 / eine Java Desktop Applikation gewählt, welche die derzeit aktiven Warnungen anzeigt.
